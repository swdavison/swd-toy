
\section{Set Up Subversion Access Control}

Source file: AiiDhAppInstSvnac.tex


As it is now set up, our Subversion repository is open to anyone who can
make an http connection to svnhost.  In this section we make configuration
changes that will limit that access.  There are two ways to do that:
\begin{itemize}
\item
We can grant read or write authorization for the entire repository 
to specific users or groups; or
\item
We can grant read or write authorization for specific directory
paths to specific users and groups.
\end{itemize}

The Red Book recommends repository-wide access control.  That is
the first method discussed below.

\subsection{Repository-Wide Access Control}


\begin{verbatim}
Edit $SWENV_PROGS/conf/httpd.conf.  Change the Location element at the 
end of the file.  The original Location element is:
<Location /awips2_repo>
   DAV svn
   SVNPath /home/swenv/swenv_progs/htdocs/awips2_repo
</Location>

The new Location element should be:
<Location /awips2_repo>
   DAV svn
   SVNPath /home/swenv/swenv_progs/htdocs/awips2_repo
  
   # How to authenticate a user
   AuthType Basic
   AuthName "AWIPS Developement"
   AuthUserFile /home/swenv/swenv_progs/conf/svn-auth-file

   # For operations other than these, require authenticated user.
   <LimitExcept GET PROPFIND OPTIONS REPORT>
      Require valid-user
   </LimitExcept>
</Location>

The new Location element tells httpd that access to anything under
awips2_repo is limited to valid users, unless that access is one of the
operations specified in the LimitExcept element.  The operations listed
in LimitExcept are Subversion operations that read the repository.
So the effect of the new Location element is to allow anyone to read
the repository while only valid users can write to it.

There are several possible schemes for repository access control.  The one
suggested here is adapted from an example in the Red Book subsection
on authorization options for the httpd server in the chapter on server
configuration.  Please see that subsection for more possibilities.

Now we must create some valid users.  Execute the commands:
    htpasswd -cm $SWENV_PROGS/conf/svn-auth-file homer
    New password: doh!
    htpasswd -m $SWENV_PROGS/conf/svn-auth-file marge
    New password: Homer!
    htpasswd -m $SWENV_PROGS/conf/svn-auth-file bart
    New password: NotMe!
After each command you will get the standard prompt to set a new password
by entering it twice.  Note that the first command has an extra option.
That option causes svn-auth-file to be created.  After executing those
commands users homer, marge, and bart will be able to commit code to
the repository.

Restart httpd:
apachectl restart

Verify that the access control is working.  Log in to some machine
with http access to svnhost.  The user name does not matter.
mkdir tmp
cd tmp
svn checkout http://svnhost:8080/awips2_repo/ade/trunk/Installer.ade

For read operations (like checkout) access is unrestricted, so the above
command should complete without a password challenge.

Now do:
cd Installer.ade
echo "#  New line" >> eclipse.sh
svn commit eclipse.sh -m"test checkin"

You will be challenged for a password.  The system will assume that
your Unix login name is your Subversion account name.  When the password
authentication fails for that user you will be prompted for a user name.
Enter homer, marge, or bart and the associated password.  The commit
should be accepted.

Subversion will save your user name and password in ~/.subversion,
and will not prompt you on subsequent commits.

Try the same operation from Eclipse.  Assuming your Eclipse workspace
is linked to the Subversion repository as described earlier, do the
following: Go to the Package Explorer view.  Navigate to the source code
file you want to check in.  Right click on that file.  Mouse over "Team"
on the context menu.  Click "Commit..." on the Team submenu.

You should get a password challenge.

\end{verbatim}

\subsection{Path-Based Access Control}

\begin{verbatim}

Depending on your situation, you may need to set up path-based
access control.  That will allow read or write access to be
granted to specific users for specific directory hierarchies
in the repository.  Please see the Red Book for a discussion of
some of the drawbacks of using path-based access control.

TBD:
I'm not sure pbac is really ready for prime time.  The
structure of the authorization file is tricky, with
authorizations being inherited from higher level directories,
over-ridden or not over-ridden by later directives, and
being set to defaults in ways I don't understand.  It may
just be that -- that I don't understand the authorization file --
or there may be bugs in the implementation.  In any case, it
seems to play especially poorly with Hudson.  

It also seems to cause checkout failures if the first item
in the set to be checked out does not require authorization.
Then no authentication challenge is given at the outset, but
that's the only time httpd will issue an authentication
challenge.  So if later in the checkout operation there is an
item that does require authorization the checkout fails 
because there is no authenticated user.

If you need that level of access control, you need to make
two changes to the setup described here:
1.  You need a different Location block.
2.  You need to install an authorization file.

The Location block would look like this:
<Location /awips2_repo>
   DAV svn
   SVNPath /home/swenv/repositories/awips2_repo

   # ----- Access control policy (authorization file)
   AuthzSVNAccessFile /home/swenv/swenv_progs/conf/svn-access-file

   # ----- First try anonymous access, identify and authenticate
   # -----    if request requires authorization.
   Satisfy Any
   Require valid-user

   # ----- How to authenticate a user.
   AuthType Basic
   AuthName "AWIPS Development"
   AuthUserFile /home/swenv/swenv_progs/conf/svn-auth-file

   # For operations other than these, require authenticated user.
   <LimitExcept GET PROPFIND OPTIONS REPORT>
      Require valid-user
   </LimitExcept>
</Location>

You can see that the Location block contains a reference to 
an "AuthzSVNAccessFile".  Please look at the file svn-access-file
in the swenv_utils.tgz tarball to see an example of the 
authorization file.  Also please read the relevant section of the
Red Book.


\end{verbatim}

