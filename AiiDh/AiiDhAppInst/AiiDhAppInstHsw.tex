

\section{Hardware, Software, Networking}

Source file: AiiDhAppInstHsw.tex

On the hardware side, the development environment consists of a single
server machine (host name svnhost in this document) and several developer
workstations.  Svnhost holds the Subversion repository and
hosts the Subversion, Hudson, and Trac server processes.

Those three processes could run on different machines.
Hudson and Trac are Subversion clients.  The machines on
which they run must be able to make an http connection with
the machine hosting the Subversion repository.  An Apache
httpd web server, version 2.x or greater, must run on the
machine hosting the Subversion repository.

If all server processes run on the same host, it must have sufficient disk
storage for: 
\begin{itemize}
\item
The AWIPS Subversion repository, which will include several
branches of the source code.  (1.5 GB on the prototype
system);
\item
Hudson's
data, which includes working copies and log files from the builds.
(7.5 GB on the prototype system);
\item
Trac's data, which will include the
issue and wiki entries and any files attached to them.  (5.5 MB on the
prototype system).
\end{itemize}

The Subversion repository takes surprisingly little space because the
source code is compressed and the various branches are stored as deltas.
As the figures for the prototype system indicate, Hudson is the heavy
hitter for disk space.  The prototype figures for Trac are probably not
useful.  In practice, Trac will probably have a many attached documents,
some of which may be quite large.  The NCEP software development managers
thought that 30 GB of available disk space would probably be adequate
for the server machine.  It will probably depend on how many old builds
Hudson is configured to store.

The OS is assumed to be RHEL 5.x, or a clone thereof.

The system is distributed in the sense that the developers may be working
on machines remote from the server.  They must have TCP/IP connections
to the server.  As described herein the two server processes, httpd and
Hudson, run on ports 8080 and 9080 respectively.

Except for the initial checkout of code, the development environment
does not demand much bandwidth.

TBD:  Installation and resource sections below specify versions.  You may
want to try different versions (already have something on your machine,
newer version available).  No reason not to do that.  But you may have
to resolve version compatibility problems.  The Apache Httpd version must
be 2.x.  The Subversion interface modules do not support Apache Httpd 1.x.

