
\section{Preliminary System Administration Tasks}


If not already installed, the following packages should be installed 
from the Red Hat repository (that is, via "yum install"):
\begin{itemize}
\item autoconf
\item gcc
\item gcc-c++
\item latex2html
\item libtool
\item libxml2-devel
\item openssl-devel
\item python-devel
\item swig
\item zlib-devel
\end{itemize}

Create user swenv as a normal login user (useradd -m swenv).  
That user may have a password or the appropriate lines may be 
added to sudoers to allow authorized individuals to su to swenv.  
Except as specifically noted, all the installation work is to be 
done by user swenv.  It could, of course, be done by the root user.  
Doing it as a non-privileged user provides some protection from 
inadvertant modification of system files.

For convenience, add the following environment variable definitions
to ~swenv/.bashrc:
\begin{verbatim}
export SWENV_PROGS=$HOME/swenv_progs
export PATH="$SWENV_PROGS/bin:${PATH}"
export DOWNLOADS=$HOME/swenv_downloads
export MYSITEPKGS=$SWENV_PROGS/lib/python2.4/site-packages
export JAVA_HOME=/awips2/java
export PYTHONPATH="$SWENV_PROGS/lib/python2.4/site-packages"
export PYTHONPATH="${PYTHONPATH}:$SWENV_PROGS/lib/svn-python"
\end{verbatim}

Download the zip files and tar balls listed in the section titled 
"References and Resources."  Put them into a directory accessible 
by user swenv.  In the rest of this document that directory will 
be referred to as \$DOWNLOADS.




