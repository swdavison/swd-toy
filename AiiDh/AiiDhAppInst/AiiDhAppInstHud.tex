
\section{Install Hudson}

Source file: AiiDhAppInstHud.tex

Hudson is a web application that integrates with Subversion to provide
the following capabilities: 
\begin{itemize}
\item	Start a build whenever code is committed to the Subversion repository.  
\item	Start periodic builds (for example, run a build at 0200 each morning).  
\item	Start a build remotely.  That is, an adminsitrator may load the 
	Hudson GUI in a web browser and start a build.  
\item	Monitor builds remotely.  That is, a user may load the Hudson
	GUI in a web browser and see the status of active or recent builds,
	including success or failure, the build log, and what prompted 
	the build (commit of a new file, manual action).  
\item	Send email notifications in the event of a build failure.
\end{itemize}

\begin{verbatim}

Install Hudson by copying its war file to the bin directory and
creating a start script:

cp $DOWNLOADS/hudson.war $SWENV_PROGS/bin/hudson.war

Create the script $SWENV_PROGS/local/bin/startHudson.sh as follows:
#!/bin/bash
export SWENV_PROGS=$HOME/swenv_progs
export HUDSON_WAR=$SWENV_PROGS/bin/hudson.war
export HUDSON_LOG=$SWENV_PROGS/logs/hudson.log
export HUDSON_HOME=$HOME/hudson_home
export HUDSON_PORT=9080
export JAVA_HOME=/awips2/java
export JAVA=$JAVA_HOME/bin/java
nohup nice $JAVA -jar $HUDSON_WAR --httpPort=$HUDSON_PORT >> $HUDSON_LOG 2>&1 &

Create the script $SWENV_PROGS/local/bin/stopHudson.sh as follows:
#!/bin/bash
kill `ps -ef | grep hudson.war | grep -v grep | awk '{ print $2 }'`

Make the new scripts executable:
chmod +x $SWENV_PROGS/bin/startHudson.sh $SWENV_PROGS/bin/stopHudson.sh

Start Hudson:
$SWENV_PROGS/bin/startHudson.sh


Verify successful installation by loading the following URL in 
a browser:
    http://svnhost:9080
You should see the Hudson "dashboard" page with the message 
"Please create new jobs to get started."

Various command line options are available for Hudson.  Please see 
the "Use Hudson" link at http://hudson-ci.org.

Install a script to rotate the Hudson log files:

    cd $SWENV_PROGS/local/bin
    tar zxf $SWENV_DOWNLOADS/swenv_utils.tgz rotateHudsonLogFiles.py
    
Add a line to the swenv crontab to run the script.  Make a
temporary copy of the swenv crontab:
    cd
    crontab -l > crontab.tmp
Edit crontab.tmp and add the following line:
49 03 * * sun $SWENV_PROGS/local/bin/rotateHudsonLogFiles.py 10 > $SWENV_PROGS/logs/rotateHudsonLogFiles.log 2>&1

Install the modified crontab file and delete the temporary copy:
    crontab crontab.tmp
    rm crontab.tmp



Set up access controls for Hudson as follows:  
    dashboard->Manage Hudson->Configure System->Enable Security->
    Hudson's own user database->Matrix-based security->
Check "Read" under "Overall" for Anonymous user.
Check "Read" and "Build" under "Job" for Anonymous user.
Add a user name, say "homer" and click all the boxes for homer
to authorize him to perform all operations.
Scroll to the bottom of the page and click "Save".

You have now authorized user homer to perform all actions, including
administrative actions.  But the user does not yet exist.  Return
to the Hudson dashboard, click "Sign Up", and fill in the form for
user homer.

Now user homer can log in and administer the Hudson installation.

Anyone who loads the top-level Hudson GUI can set up an account, but
the new users will only have the priveleges of the anonymous user until
user homer (or some other user with administrative privileges) goes to
the configuration screen:
    dashboard->Manage Hudson->Configure System
adds that user and authorizes that user for specific operations.

Hudson has its own user data base separate from Unix or httpd.
Only administrators (those who will be setting up new Hudson
builds) will need Hudson accounts.  Developers can use Hudson to 
check the status of builds without logging in.

Working as user homer or some other user with appropriate authorization,
carry out the following steps to set up the build.  Please see the Hudson
documentation referenced above for other ways to do it.  Where a GUI
navigation sequence starts with "Hudson", start from the main Hudson
dashboard (http://svnhost:9080).

Set Hudson to run only one build at a time.  We do this because we will
have several jobs using the same workspace and they might otherwise
interfere with each other.

Hudson -> Manage Hudson -> Configure System -> 
Set "# of executors" to 1.  Click "Save" at the bottom of the GUI.

Set up a job to build CAVE.  We will build it from the developer 
branch of SVN.
Hudson -> New job
Set Job name CAVE_developer.
Select "Build a free-style software project.
Click OK.
Return to the Hudson dashboard.

Hudson -> CAVE_developer -> Configure
Select  Subversion under Source Code Management
Enter the string "http://svnhost:8080/awips2_repo/awips/branches/developer"
in the repository URL keyin.  
Under "Build Triggers" select "Poll SCM."  Do not enter anything 
in the "Schedule" text field that appears.

Under "Build" click "Add build step" and then "Execute shell."  Enter 
the following shell commands in the text field that appears:
    cd $WORKSPACE/developer/build
    export PATH=/awips2/java/bin:${PATH}
    sh build.sh -eclipse=$WORKSPACE/developer/uframe-eclipse

Click "Save" at the bottom of the GUI.

We are now set up for manual (that is, non-automated) builds via 
Hudson.  There are three ways to start a build.
Click "Build now" on Hudson's CAVE_developer project page.
Load the URL http://svnhost:9080/job/CAVE_developer/build
Load the URL http://svnhost:9080/job/CAVE_developer/polling

Click "Build now."  A box labeled "Build History" appears with a date,
time, and build number.  Click on the date and time to get to the
build information page.  Then click on "Console Output." You will see
Subversion output indicating that the source code is being checked out
by Hudson in preparation for the build.

Create a Hudson job to build EDEX:
Hudson -> New Job
Job name: EDEX_developer
Select "Copy existing job", Copy from CAVE_developer.
OK
Return to the Hudson dashboard.

Hudson -> EDEX_developer -> Configure -> Advanced Project Options ->
Use Custom Workspace.
Directory:  $HUDSON_HOME/jobs/CAVE_developer/workspace

Under "Build" in the "Execute shell" text input area, change the first 
command (the cd command) to:
    cd $WORKSPACE/developer/build.edex
Click Save at the bottom of the GUI.

Create a Hudson job  to build RCM:
Hudson -> New Job
Job name: RCM_developer
Select "Copy existing job", Copy from EDEX_developer.
OK
Return to the Hudson dashboard.

Hudson -> RCM_developer -> Configure
Under "Build" in the "Execute shell" text input area, change the 
first command (the cd command) to:
    cd $WORKSPACE/developer/build.rcm
Click Save at the bottom of the GUI.



	SUBSECTION:  Setting up Automated Builds in Hudson

TBD:  Don't do it this way.  We do not want Hudson starting a
build after every checkin.  Instead, set it up so that Hudson
checks Subversion for updates periodically (every hour, say)
and runs a build if there has been a change since the last
build.


We use Subversion's "post commit hooks" feature to set up 
automated builds in Hudson.

cd ~/repositories/awips2_repo/hooks
cp post-commit.tmpl post-commit
chmod +x post-commit

Edit file post-commit.  Delete the mailer.py command at the bottom
and add the commands:
    wget -q -O /dev/null http://svnhost:9080/job/CAVE_developer/polling
    wget -q -O /dev/null http://svnhost:9080/job/EDEX_developer/polling
    wget -q -O /dev/null http://svnhost:9080/job/RCM_developer/polling

The wget command uses the http protocol to fetch a file.  In this case
the effect (on Hudson) is the same as loading the URL in a web browser.
Since we are using the "polling" target, Hudson will not do a build
unless there have been source code changes since the last build.  So,
for example, if code has been committed to the trunk of the Subversion
repository the post-commit script will be executed.  But Hudson will
find no changes in the developer branch, which is the one linked to the
CAVE_developer, EDEX_developer, and RCM_developer jobs.

If we want Hudson to kick off a CAVE build after code is committed to the
trunk, we can create a new Hudson job called (for example) CAVE_trunk
and link it to the trunk of the Subversion repository.  Then we could
add the line:
    wget http://svnhost:9080/job/CAVE_trunk/polling
to the list of commands above.

After any commit to any part of the repository all the wget commands in
post-commit get executed.  But because we are using the URL for polling,
Hudson will only execute a build if the relevant part of the repository
has changed.

Verify that the automated build system works.  Get the Hudson dashboard
visible in a browser window.  If there is a link "ENABLE AUTO REFRESH"
in the upper right corner click on it.  Then check out some code, alter
it, and check it back in:

    mkdir tmp
    cd tmp
    export ROOT=http://svnhost:8080/awips2_repo/awips/branches/developer
    export DIR1=com.raytheon.uf.viz.core/src/com/raytheon/uf/viz/core
    export DIR2=localization
    export URL=$ROOT/$DIR1/$DIR2
    svn checkout $URL
    cd localization
    echo "// New line in the file" >> ConnectivityPreferenceDialog.java
    svn commit ConnectivityPreferenceDialog.java -m"testing autobuild"

You will see the Hudson dashboard show that a new build is in progress.
If you navigate to the build page you can see that it was a change to
ConnectivityPreferenceDialog.java that caused the build to start.

\end{verbatim}

