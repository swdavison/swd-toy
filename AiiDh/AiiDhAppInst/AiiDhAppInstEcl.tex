
\section{Connect to Repository from Eclipse and Check Code Out}

Source file: AiiDhAppInstEcl.tex

\begin{verbatim}

TBD:
o  Rewrite this section.  Initial checkout of the whole code base,
   all the Eclipse projects, is from the trunk.  Note
   that initial checkout might be via removable media if the connection
   from the workstation to the Subversion machine is not very fast:
   Check out trunk to a temp dir on a thumb drive on the Subversion 
   machine.  Developer copies it to his Eclipse workspace.  Then
   developer creates a branch only of the Eclipse project he is 
   working on, deletes the trunk checkout of that project and checks
   out the project from the branch.  Or uses svn switch to change
   that project from a trunk working copyt to a branch working copy.
   All that needs to be tested.
o  Move this section out of the installation appendix and into 
   "Using the SED" section.


The version of Eclipse delivered as part of the ADE includes the Subclipse
plugin for working with Subversion.  Subclipse provides Eclipse GUIs that
support fetching code from and committing code to Eclipse repositories.
The following procedure describes how to bind to a Subversion repository,
fetch code, and commit code.  It is assumed that you have installed the
ADE as described in the flow tag.  Any user with a network connection
to svnhost can carry out this procedure.


Start Eclipse
Deselect "Build Automatically" in the Project menu.
Window->Open Perspective->Other...->SVN Repository Exploring->OK
Right-click in the SVN Repositories window (which will be empty at first).
In the context menu that pops up, select New->Repository Location...
In the keyin labeled URL enter "http://svnhost:8080/awips2_repo.
Click Finish.

The repository URL you entered will appear in the SVN Repositories window.
Expand awips through its various levels until you have expanded the
"developer" level.  
Click on the first entry under "developer" ("build")
to select it.  
Scroll down and shift-click on the last entry under
"developer" (uk.ltd.getahead) so that all the entries under "developer"
are selected.  
Right click on any one of the selected entries and select
"Checkout..." from the context menu that pops up.  
Accept all defaults
on the "Checkout from SVN" dialog that appears.

\end{verbatim}


