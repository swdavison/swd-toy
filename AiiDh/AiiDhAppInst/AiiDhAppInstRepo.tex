
\section{Create Subversion Repository and Load AWIPS II Code}

Source file: AiiDhAppInstRepo.tex


\subsection{Prepare AWIPS Source Code for Import into Subversion
Repository}

Follow the directions in the ADE setup flow tag through the step
where AWIPS II is built.  In the r1g1-4 ADE flow tag that step
is on page 12 of the PDF file and is titled ``Compile AWIPS II.''
Verify that the build completes successfully as described in the
flow tag.  Then make sure that ``build automatically'' is turned off
in the project menu and select ``Clean...'' from the project menu.
Be sure that ``Clean all projects'' is selected and 
"Start a build immediately" is not checked on the Clean
dialog.

Purge from the projects directories files that we do not want to import
into the Subversion repository:

\begin{verbatim}
    export PRJ_SRC=${HOME}/ade/projects
    cd $PRJ_SRC
    rm -rf javadocs
    find ade awips Installer Installer-bundle-example -type d -name .svn \
        -exec rm -rf {} \;
\end{verbatim}


\subsection{Repository Structure}

There are various possible conventions for setting up
the repository.  See page 123 of the Red Book.  See also
http://www.awips2omaha.com/trac/awips/wiki/Configuring%20Subclipse for
an indication of how the Raytheon/Omaha development team structures
their repository.  In our case the repository structure will look like:

\begin{verbatim}
    awips2_repo/
        ade/
            trunk/
            tags/
            branches/
        awips/
            trunk/
            tags/
            branches/
        Installer/
            trunk/
            tags/
            branches/
        Installer-bundle-example/
            trunk/
            tags/
            branches/
\end{verbatim}

The ade, awips, Installer, and Installer-bundle-example are the
projects from ade/projects/.  The next-level directories are:
\begin{itemize}
\item
trunk.  Reviewed and tested code.  Modified as new tasks are delivered.
\item
tags.  Snapshots.  Created for releases or other events where it is
convenient to have a pointer to the state of the repository at that
time.  Once created a tag should never be modified.
\item
branches.  Branches for individual developers will be created under
this directory.  Typically a branch is created by an individual 
developer for a specific task.  Branches are usually copied from the
trunk.  After the task is completed the branch is merged back into the
trunk.  That merge constitutes delivery of the code.  The branch is 
then deleted.
\end{itemize}


\subsection{Import the Source Code}

The procedure below is for initial import of the source code. 
Please see the section titled ``Administering the Software
Development Environment'' for a procedure for merging 
Raytheon deliveries after the first one into the repository.

\begin{verbatim}
    export REPO_DIR=$HOME/repositories
    mkdir $REPO_DIR
    cd $REPO_DIR
    svnadmin create $REPO_DIR/awips2_repo
    export PRJ_SRC=$HOME/ade/projects
    cd $PRJ_SRC
    export URL=file:///home/swenv/repositories/awips2_repo
    svn import --no-ignore -m "New import" ade $URL/ade/trunk
    svn import --no-ignore -m "New import" awips $URL/awips/trunk
    svn import --no-ignore -m "New import" Installer $URL/Installer/trunk
    svn import --no-ignore -m "New import" Installer-bundle-example \
        $URL/Installer-bundle-example/trunk
\end{verbatim}

We use the ``no-ignore'' option to svn import because there are certain
file types that Subversion (by default) treats as build products, and
therefore "ignores,"  that is, does not import them into the repository.
We need those build products (third party jar files, for example)
because they constitute part of the AWIPS configuration.

Check success with
\begin{verbatim}
    svn list --depth infinity --verbose $URL
\end{verbatim}
or
\begin{verbatim}
    svnlook tree ~/repositories/awips2_repo 
\end{verbatim}

In our work flow and code life cycle processes, each developer will have
his or her own branch of the repository.  They will check code in and
out of that branch and deliver it to the trunk only after review and
acceptance.  Our next step is to create a developer branch for testing
the installation.


\begin{verbatim}
    export URL=file:///home/swenv/repositories/awips2_repo
    svn copy $URL/ade/trunk $URL/ade/branches/developer --parents \
        -m "Creating developer branch"
    svn copy $URL/awips/trunk $URL/awips/branches/developer --parents \
        -m "Creating developer branch"
    svn copy $URL/Installer/trunk $URL/Installer/branches/developer --parents \
        -m "Creating developer branch"
    svn copy $URL/Installer-bundle-example/trunk \
        $URL/Installer-bundle-example/branches/developer --parents \
        -m "Creating developer branch"

Now we will make the repository visible to the Apache httpd web server,
so that remote users can view it in web browsers and so that httpd can
be used as the Subversion server.

Make a symbolic link in the htdocs directory pointing to the repository:
    ln -s ~swenv/repositories/awips2_repo $SWENV_PROGS/htdocs/awips2_repo

Add the following lines to the end of $SWENV_PROGS/conf/httpd.conf:
<Location /awips2_repo>
   DAV svn
   SVNPath /home/swenv/swenv_progs/htdocs/awips2_repo
</Location>

Start or restart httpd with
    $SWENV_PROGS/bin/apachectl start
or 
    $SWENV_PROGS/bin/apachectl restart

Open a browser and load the URL
    http://svnhost:8080/awips2_repo
You should now be able to navigate the repository and view 
files in the browser.

You should also be able to use the http protocol to check code out:
    mkdir tmp
    cd tmp
    svn checkout http://svnhost:8080/awips2_repo/ade/trunk/Installer.ade


\end{verbatim}

