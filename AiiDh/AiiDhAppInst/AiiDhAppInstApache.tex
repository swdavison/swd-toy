
\section{Install Apache httpd}

Source file: AiiDhAppInstApache.tex

We use the Apache httpd web server as both the Subversion server
and the Trac server.  Note that we must use Apache 2 or
later.  There are no Subversion plugins for the Apace 1
series.

\begin{verbatim}

    export SWENV_PROGS=$HOME/swenv_progs
    mkdir $SWENV_PROGS
    cd $SWENV_PROGS
    mkdir bin  etc include lib libexec sbin share src

    export TMP_BLD_DIR=${HOME}/tmpblddir
    mkdir $TMP_BLD_DIR

    cd $TMP_BLD_DIR
    tar zxf $DOWNLOADS/httpd-2.2.15.tar.gz
    cd httpd-2.2.15
    ./configure --prefix=$SWENV_PROGS --with-included-apr --enable-rewrite \
        --enable-modules=most --enable-mods-shared=most --enable-ssl  \
        --enable-dav --enable-dav-fs --enable-dav-lock --with-port=8080 \
        --with-sslport=8443 \
         >& configure.out
    make >& make.out
    make install >& makeinstall.out

    cd
    mv $TMP_BLD_DIR tmpblddir.httpd

\end{verbatim}

Edit \$SWENV\_PROGS/conf/httpd.conf and make the following
changes:
\begin{itemize}

\item
Uncomment the ServerName line and add
a host name or IP address in accordance with the comment in httpd.conf.
If the server has a registered DNS name the ServerName line will look
like:
\begin{verbatim}
    ServerName www.example.com:8080
\end{verbatim}
If the server does not have a registered domain name use the IP address.
The ServerName line will look like:
\begin{verbatim}
    ServerName 123.45.67.89:8080
\end{verbatim}

\item
Change the CustomLog and ErrorLog lines in httpd.conf to
give the log files names indicating their origin.  The new 
lines should be:
\begin{verbatim}
    CustomLog "logs/access_log.httpd" common
    ErrorLog "logs/error_log.httpd"
\end{verbatim}

\end{itemize}

Verify successful installation and startup.  Execute the command:
\begin{verbatim}
    $SWENV_PROGS/bin/apachectl start
\end{verbatim}
On a machine with a network connection to svnhost, load the URL
http://svnhost:8080 into a browser.  You should see a web page with the
text "It works!"

You will probably want to have Apache httpd start up automatically
when the server machine is rebooted.  That must be set up by the
root user.  Please see the section titled "Final System Administration
Tasks"

The logs must be rotated and purged.  One common way to do
that is to use the system logrotate facility.  That must be
set up by the root user.  If you plan to do it that way,
please see the man page for logrotate.  

To keep the software development environment as
self-contained as possible, to minimize intrusion on system
files and directories, and to minimize activities that must
be performed by the root user we use a different method.
A line in the crontab file for user swenv
periodically runs a script to rotate the log files.  Install
the rotation script with the following commands:
\begin{verbatim}
    mkdir -p $SWENV_PROGS/local/bin
    cd $SWENV_PROGS/local/bin
    tar zxf $SWENV_DOWNLOADS/swenv_utils.tgz rotateHttpdLogFiles.py
\end{verbatim}

To add the new cron, make a temporary copy of the crontab file with 
the command
\begin{verbatim}
    crontab -l > crontab.tmp
\end{verbatim}
Add a line like this at end of the file:
\begin{verbatim}
    47 03 * * sun $SWENV_PROGS/local/bin/rotateHttpdLogFiles.py 10
\end{verbatim}

Install the modified crontab file and delete the temporary copy:
\begin{verbatim}
    crontab crontab.tmp
    rm crontab.tmp
\end{verbatim}

That line will make cron run the rotateHttpdLogFiles.py
script every Sunday morning at 03:47.  The script will keep
ten compressed copies of the log files.  


