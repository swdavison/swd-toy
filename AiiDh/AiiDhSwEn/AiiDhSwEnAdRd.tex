
\subsection{How to Merge a New Raytheon Delivery into the Repository}

Source file: AiiDhSwEnAdRd.tex

From time to time the AWIPS main contractor delivers a
new version of the software.
There will be files that have been modified both by the
contractor and by other developers.  We want to minimize the
work required to resolve conflicts and merge those files.
In particular, we do not want to have to merge the same set
of changes repeatedly, at each new delivery.

The Red Book discusses that problem in the section titled
``Vendor Branches.''  The procedure presented here follows that 
approach.  If you don't know what ``branch,'' ``trunk,'' and ``tag''
mean in the context of Subversion you should probably read 
the section of this document titled ``Working with Subversion''
before reading this section.

Here is a high-level view of the procedure:
\begin{enumerate}
\item
We maintain what amounts to a separate trunk for the
vendor.  In the Red Book description and in the discussion
that follows it is called the ``current'' branch.
\item
With each vendor delivery we update the current branch to 
reflect that delivery.  Note that the current branch only
contains changes made by the vendor.
\item
With each vendor delivery we create a tag from the current
branch: a snapshot of the branch after that delivery.
\item
With each vendor delivery we use Subversion's merge facility
to merge into a working copy of our trunk the changes the 
vendor has made since the previous delivery.
\item
After resolving any conflicts and performing all regression
tests we commit the working copy back to the trunk.  
\end{enumerate}

Consider the repository structure shown below.  It is
similar to the structure shown in the section titled
``Create Subversion Repository and Load AWIPS II Code''
but it contains a new high-level directory called
``vendor.''  The vendor directory contains a subdirectory
for each project we have under version control.  Each of
those projects contains a subdirectory called ``current.''
The current directory is similar to the trunk directory,
except that it contains only changes made by the vendor, not
changes made by any other development organization.

The other directories at the same level as current (r1g1 and
r1g1-4 in the example below) are tags.  While the contents
of the current directory change with each release, the other
directories (r1g1, r1g1-4, etc.) are static, reflecting the
contents of specific deliveries.  For each new delivery we
will get another such directory tagged with that delivery's
name.

The example structure below shows a situation in which the 
first vendor delivery was r1g1 and the second was r1g1-4.
The example reflects the situation after integration of the 
r1g14 delivery.

\begin{verbatim}
    awips2_repo/
        vendor/
            ade/
                current/
                r1g1/
                r1g1-4/
            awips/
                current/
                r1g1/
                r1g1-4/
            Installer/
                current/
                r1g1/
                r1g1-4/
            Installer-bundle-example/
                current/
                r1g1/
                r1g1-4/
        ade/
            trunk/
            tags/
            branches/
        awips/
            trunk/
            tags/
            branches/
        Installer/
            trunk/
            tags/
            branches/
        Installer-bundle-example/
            trunk/
            tags/
            branches/
\end{verbatim}


\subsubsection{Create Current Branch and First Tag}

You might do this step when you import the first delivery (r1g1
in this example) into the repository, or you might do it later.
If you do it later (as this example assumes) you must identify
the repository revision number at which the initial vendor delivery
had been imported but no local changes had been made.  In this
example we assume that is revision four.

Do this for each of the four projects in the ade/projects
directory: ade, awips, Installer,
Installer-bundle-example.  In this example we step through the
procedure for the ade project.

Create the original current branch and the r1g1 tag for the ade project.
Both will be copies of r4 of the corresponding trunk.  This is a one-time
operation.  It will not be repeated for new deliveries from Raytheon.

\begin{verbatim}
export URL=http://swdavison.dyndns.org:8080/awips2_repo
svn copy -r 4 $URL/ade/trunk $URL/vendor/ade/current --parents \
    -m "Creating current branch"
svn copy -r 4 $URL/ade/trunk $URL/vendor/ade/r1g1 --parents \
    -m "Creating r1g1 tag"
\end{verbatim}


\subsubsection{Update Current Branch with New Delivery and Create Tag}

Update the ``current'' branch for ade with the ade project
from r1g1-4 and create the r1g1-4 tag.  
Use the svn\_load\_dirs.pl
helper script for this as described in the Red Book.  In the 
commands below the environment variable R1G14 points to a ``cleaned''
version of the ade/projects/ade directory as installed in 
accordance with the ADE flowtag.  (By ``cleaned'' I mean that 
we have removed files and directories we don't want under version
control, like the .svn directories.)

\begin{verbatim}
export URL=http://swdavison.dyndns.org:8080/awips2_repo
export SLDDIR=/home/swenv/swenv_progs/contrib/client-side/svn_load_dirs
export SLD=$SLDDIR/svn_load_dirs.pl
export R1G14=$HOME/ade.r1g1-4.cleaned
$SLD -t r1g1-4 $URL/vendor/ade current $R1G14
\end{verbatim}

\subsubsection{Prepare for the Merge}

Before proceeding, make sure that the trunk is up to date.
That is, reintegrate developer branches into the trunk.
(See the section titled ``How To Deliver Code to the Trunk.'')
Developers who do not reintegrate their branches into the trunk
at this stage will have a lot more work to do to deliver their
code later on.

As usual when reintegrating development branches, delete them after
the reintegration.

\subsubsection{Merge Vendor Changes into Working Copy of Trunk}

Make a working copy of the trunk.

\begin{verbatim}
svn checkout $URL/ade/trunk trunk_wc
\end{verbatim}

Merge the differences between the r1g1 tag and the current branch
(which is up to date with r1g1-4) into the working copy of the trunk.

\begin{verbatim}
svn merge    $URL/vendor/ade/r1g1     \
             $URL/vendor/ade/current  \
             trunk_wc
\end{verbatim}

\subsubsection{Resolve Conflicts, Test, Commit}

Resolve any conflicts between their changes and our changes.
This, of course, is the hard part.  Test to make sure everything 
works.

Finally, commit the working copy back into the trunk.

\begin{verbatim}
cd trunk_wc
svn commit -m"Merging Raytheon delivery r1g1-4 into trunk"
\end{verbatim}



