
\subsection{Working with Subversion}

Source file: AiiDhSwEnUsSu.tex

Subversion is our version control system.  It is a widely
used free, open-source package.  There is lots of good
documentation for Subversion:
\begin{itemize}
\item
If you simply type ``svn'' at the command line you will get
usage information leading you into the help facility that
gets installed with Subversion.
\item
There is a good book available: ``Version Control with
Subversion.''  You can buy a printed copy of the book,
access an HTML version on line, or download it as a PDF to
refer to or print out.  In this Handbook and in web pages
discussing Subversion it is often referred to as the ``Red Book''.
Please see the section of this Handbook 
titled ``Appendix \ref{ReferencesAndResources}: References and Resources.''
for full bibliographic information and URLs.
\item
Because Subversion is so widely used you can often get a
quick answer to a question by Googling.
\end{itemize}

This section of the Handbook provides an overview of
Subversion.  It is not intended to substitute for any of
the information sources listed above.  You will find it
well worth while to
become familiar with those sources.  

Other subsections of
the Handbook will provide cookbook-type examples of how to
accomplish tasks that may crop up frequently in the AWIPS
development environment.  What follows in this section is
a quick, high-level summary to give you some context when
you read the ``How To'' sections for specific tasks.

We can think of Subversion as consisting of three
components:
\begin{itemize}
\item
Data base files containing the source code that Subversion is
keeping track of.  Those files are on the server machine.  It
is best to consider those files a black-box implementation 
detail.  There will be little further discussion of them in
this document.  It is useful to remember, however, that a
source code file under Subversion configuration management
is not associated with any particular file in the Subversion
repository.  It is a set of records in one or more data
base cache files.
\item
Various server-side utility programs and API-accessible
functions.  The Subversion API is out of scope for this
document.  Svnadmin and svnlook are examples of server 
side utilities.  You can find examples of their usage in
the section titled ``Install Subversion,'' but they are
rarely needed by ordinary users and will not be discussed
further in this section.
\item
A server program.  Discussed in this section.
\item
One or more client programs.  Those must be installed on
the user's workstation.  Discussed in this section.
\end{itemize}


\subsubsection{Server}

Subversion comes with its own server (called svnserve),
but we do not use that.  Instead, we use the Apache web
server (httpd) as the subversion server. Httpd is
configured to call Subversion plugins when you access the
Subversion repository via http.  For example, if you enter
the following URL into a web browser you will see the
AWIPS2 repository:
\begin{verbatim}
    http://swdavison.dyndns.org:8080/awips2_repo
\end{verbatim}


\subsubsection{Client}

The AWIPS project uses several client programs for Subversion:
\begin{itemize}
\item
As indicated above, any web browser can serve as a
subversion client.
It provides a quick
and easy way to access the ``head'' revision of the software
under control.  (The head revision is the most recent
revision.)  But that's all you can do with the web
browser.  With the web browser you cannot edit code and
you cannot see old versions.
\item
Trac incorporates a Subversion client.  At the top of most
Trac web pages you will see a button labeled ``Browse
Source.``  That will take you into the repository.  It
only provides read access, but you can see old revisions,
commit comments, and other information.
\item
The version of Eclipse delivered by Raytheon as part of
the ADE includes a Subversion client packaged as a plugin
called Subclipse.  That allow you to use the Eclipse GUI
to access the repository, including checking code in and
out and other capabilities.
\item
Subversion provides a command line client called svn.  You
will see examples of svn usage in the rest of this section.
\end{itemize}


\subsubsection{Checkout and Commit (Checkin)}

Unlike other commonly used revision control systems like
CVS or RCS, Subversion does not (in the course of routine
usage) lock source code files.  If you execute the following
commands:
\begin{verbatim}
    mkdir ~/workarea.tmp
    cd ~/workarea.tmp
    export REPO_URL=http://swdavison.dyndns.org:8080/awips2_repo
    export HBK_URL=$REPO_URL/AiiDh/branches/sandbox
    svn checkout $HBK_URL
\end{verbatim}

you get a copy of the source code for this Handbook in the
workarea.tmp/sandbox directory.  In Subversion documentation
the workarea.tmp/sandbox directory is called a ``working
copy.''
On the server side, the code is
not locked.  The only record of the access will be in the
Apache httpd log.  Subversion will keep no record of it.
Other developers can check it out without any difficulty,
and without any notification that you have checked it out.
This is an intentional design feature of Subversion.  The
reasoning behind it is discussed in the Red Book.

(The checkout procedure
described above involves a lot of typing.  In actual
routine usage, as those experienced with development on
Unix platforms are aware, most of that typing would be
taken care of by environment variable and alias
definitions in the developer's environment setup files.)

Now you can edit the source code files for the handbook and,
when you are finished, you can check them in:
\begin{verbatim}
    svn commit -m"Example"
\end{verbatim}
(The above command assumes that your CWD is
~/workarea.tmp/sandbox.)


\subsubsection{Working Copies}

You will notice that the commit command did not contain
the repository URL.  How did svn know where to commit the
changes?  The answer is that the sandbox subdirectory,
created by the checkout command, contains metadata files
in a subdirectory called .svn.  Every source code directory
and subdirectory in a working copy contains a .svn
subdirectory.  Files in that directory contain 
information about the repository from which
the working copy was checked out.

One consequence of this arrangement is that only
directories, not individual files, can be checked out.
Individual files can be committed.

You can go ahead and execute the checkout and commit commands
above if you are interested.  The ``sandbox'' repository
directory, as its name indicates, is provided for
experimentation.  If it gets too messed up Subversion 
provides easy ways to restore it to a sane condition.

\subsubsection{Branches}
In Subversion jargon, the sandbox repository directory is 
called a branch.  In the AWIPS development environment
every developer working on a project has his or her
own branch.  The Subversion name of the Handbook project
is AiiDh.  That project has a subdirectory called branches.
Every developer working on the Handbook will have
a branch under that.  Indeed, you
might have more than one branch if you are working on
several tickets and want to keep the work separate.

Since Subversion stores all branches by deltas
(differences) they do not take up much space.  Since you
have your own branch, you can feel free to check your code
in whenever you choose without worrying about breaking
the build for other developers.  That way you have access to all your old
versions as well as taking advantage of the main
repository's backup facilities.


\subsubsection{Trunk}

Most repository projects contain a subdirectory called
``trunk''.  The trunk subdirectory contains reviewed,
``official'' code.  Branches are created by making copies
of the trunk, or of other branches.
Please be sure not to check code in to
the trunk without going through the proper reviews first.

If you accidentally check code into the trunk, please
notify the Repository Manager or the Development Manager
right away.  It's very easy to ``revert'' the trunk back to
its earlier state, so such accidents are not big deals as
long as they are promptly corrected.
But if they are left uncorrected then other changes may be
checked in on top of them.  After that has happened it's a
lot more work to disentangle the intentional and the
accidental updates.

As described above, when you commit code back to the
Subversion repository it goes back in to the part of the
repository from which it was checked out.  Since you
should not routinely check code in to the trunk, you
should avoid checking code out of the trunk.  If you want
a copy of the code in the trunk you should create a branch
and check the code out of that branch.  See the ``How To''
sections of this manual for a description of how to do
that.

Remember, by the way, that if you just want to look at the
code in the trunk you can do that through the read-only
interfaces provided by the web browser and by Trac.


\subsubsection{The Svn Client}

The svn client utility has a large number of subcommands.  The
Red Book has a reference section describing all of them.  Two
very useful subcommands are ``info`` and ``status``.

If you type ``svn info'' in a subdirectory created by an 
``svn checkout'' command, svn will look in the metadata directory 
(.svn) and provide information about the working copy,
including the repository branch from which it came.

If you type ``svn status'' in a Subversion working copy
directory, you will see a list of some of the files in the
directory.  Here is an example:
\begin{verbatim}
    $ svn status
    ?       AiiDh.out
    ?       AiiDh.pdf
    ?       globalRevisionNumber.tex
    ?       builddate.tex
    ?       AiiDh.log
    ?       AiiDh.toc
    ?       AiiDh.aux
    ?       AiiDh
    M       AiiDh.tex
\end{verbatim}

The files flagged by question marks are "unversioned files."  Those
are files that have never been checked in to subversion.  If you do
an ``svn commit'' operation those files will not be checked in.  In
this case we would not want those files to be checked in: They are
build products created by the build script and pdflatex processors
when the Handbook was built from source.  If you create a new file
that you want checked in, you must use the ``add'' subcommand.  For 
example:
\begin{verbatim}
    svn add NewClass.java
\end{verbatim}
That command will make svn update the metadata so that when you
later do an ``svn commit'' command the new Java source code file
will be added to the repository.

The file flagged with an ``M'' above has been modified since it
was checked out.  When you commit the directory the new version of
the file will go into the repository.


\subsubsection{Warning: Subversion Never Forgets}

Subversion never forgets.  That's good, because it makes it easy
to correct most accidental commit actions.  But it is {\em very} difficult
to {\em completely} remove anything from a Subversion repository.
There is a delete command.  That command creates a new version of the
repository without the deleted item.  But the item is still there in
previous versions of the repository.  So please take care
never to commit documents that should be private (password
files, personal files you do not want others to read, etc.)

Subversion stores source code files in compressed form as
data base entries.  The data base schema is a "black box" in
the sense that it is an implementation detail with no
official documentation.  Removing a file (or data from a
file)  completely would
require using SQL commands to delete certain
fields or rows, and updating whatever other tables might
refer to them.

It is very likely that almost all of the AWIPS code base will 
eventually be
released for public use.  But AWIPS does contain some restricted
code that the Government is permitted to use only under specific
licensing agreements.  We must take care not to publish any
repository which has ever held any of that code.


