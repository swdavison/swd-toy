
\chapter{Software Development Environment}

The software development environment (SDE) is a set of tools to 
minimize the burden imposed on developers and management by 
the software development process.  In the context of this 
document the SDE is the set of 
tools (a software ``stack'') used by one or more software 
development teams.  The AWIPS II development environment for 
individual programmers is defined by the AWIPS Development 
Environment (ADE) package delivered and documented by Raytheon 
as part of the AWIPS Migration project.

On the server side, the SDE consists of:
\begin{itemize}
\item
Subversion.  The source code management system.
\item
Trac.  The wiki and issue tracking system.
\item
Hudson.  The build management system.
\end{itemize}

Please see the section titled ``References and Resources''
for links and references to more information on those 
server-side components.  The section titled
``Installing the Software Development Environment''
describes how to install them.

On the client side, the SDE consists of:
\begin{itemize}
\item
Firefox.  Any other web browser should work.  The web
browser serves as the client for the Subversion, Trac, and
Hudson servers.
\item
ADE.  This is the AWIPS Development Environment, which is
delivered with each AWIPS release by Raytheon.  It includes
Eclipse.
\item
Svn.  This is the Subversion command line client.  The
version of Eclipse included with the ADE has a built-in
Subversion client (the Subclipse plugin), but users who want
to have the full power of Subversion available will want to
use svn.  Svn is also very well documented in the ``Red
Book.''  (Please see the section titled ``References and
Resources'' for a link to that book.)
\end{itemize}



\section{Administering the Software Development Environment}

Intended for CM, development, build managers.  How to set up 
authorization/authentication for various parts of the SDE.
How to Merge a New Raytheon Delivery into the Repository

Discuss synchronization of repositories.  (Not really possible w/
Subversion)

Detailed.  Low-level.

% ----- How to Merge a New Raytheon Delivery into the Repository

\subsection{How to Merge a New Raytheon Delivery into the Repository}

Source file: AiiDhSwEnAdRd.tex

From time to time the AWIPS main contractor delivers a
new version of the software.
There will be files that have been modified both by the
contractor and by other developers.  We want to minimize the
work required to resolve conflicts and merge those files.
In particular, we do not want to have to merge the same set
of changes repeatedly, at each new delivery.

The Red Book discusses that problem in the section titled
``Vendor Branches.''  The procedure presented here follows that 
approach.  If you don't know what ``branch,'' ``trunk,'' and ``tag''
mean in the context of Subversion you should probably read 
the section of this document titled ``Working with Subversion''
before reading this section.

Here is a high-level view of the procedure:
\begin{enumerate}
\item
We maintain what amounts to a separate trunk for the
vendor.  In the Red Book description and in the discussion
that follows it is called the ``current'' branch.
\item
With each vendor delivery we update the current branch to 
reflect that delivery.  Note that the current branch only
contains changes made by the vendor.
\item
With each vendor delivery we create a tag from the current
branch: a snapshot of the branch after that delivery.
\item
With each vendor delivery we use Subversion's merge facility
to merge into a working copy of our trunk the changes the 
vendor has made since the previous delivery.
\item
After resolving any conflicts and performing all regression
tests we commit the working copy back to the trunk.  
\end{enumerate}

Consider the repository structure shown below.  It is
similar to the structure shown in the section titled
``Create Subversion Repository and Load AWIPS II Code''
but it contains a new high-level directory called
``vendor.''  The vendor directory contains a subdirectory
for each project we have under version control.  Each of
those projects contains a subdirectory called ``current.''
The current directory is similar to the trunk directory,
except that it contains only changes made by the vendor, not
changes made by any other development organization.

The other directories at the same level as current (r1g1 and
r1g1-4 in the example below) are tags.  While the contents
of the current directory change with each release, the other
directories (r1g1, r1g1-4, etc.) are static, reflecting the
contents of specific deliveries.  For each new delivery we
will get another such directory tagged with that delivery's
name.

The example structure below shows a situation in which the 
first vendor delivery was r1g1 and the second was r1g1-4.
The example reflects the situation after integration of the 
r1g14 delivery.

\begin{verbatim}
    awips2_repo/
        vendor/
            ade/
                current/
                r1g1/
                r1g1-4/
            awips/
                current/
                r1g1/
                r1g1-4/
            Installer/
                current/
                r1g1/
                r1g1-4/
            Installer-bundle-example/
                current/
                r1g1/
                r1g1-4/
        ade/
            trunk/
            tags/
            branches/
        awips/
            trunk/
            tags/
            branches/
        Installer/
            trunk/
            tags/
            branches/
        Installer-bundle-example/
            trunk/
            tags/
            branches/
\end{verbatim}


\subsubsection{Create Current Branch and First Tag}

You might do this step when you import the first delivery (r1g1
in this example) into the repository, or you might do it later.
If you do it later (as this example assumes) you must identify
the repository revision number at which the initial vendor delivery
had been imported but no local changes had been made.  In this
example we assume that is revision four.

Do this for each of the four projects in the ade/projects
directory: ade, awips, Installer,
Installer-bundle-example.  In this example we step through the
procedure for the ade project.

Create the original current branch and the r1g1 tag for the ade project.
Both will be copies of r4 of the corresponding trunk.  This is a one-time
operation.  It will not be repeated for new deliveries from Raytheon.

\begin{verbatim}
export URL=http://swdavison.dyndns.org:8080/awips2_repo
svn copy -r 4 $URL/ade/trunk $URL/vendor/ade/current --parents \
    -m "Creating current branch"
svn copy -r 4 $URL/ade/trunk $URL/vendor/ade/r1g1 --parents \
    -m "Creating r1g1 tag"
\end{verbatim}


\subsubsection{Update Current Branch with New Delivery and Create Tag}

Update the ``current'' branch for ade with the ade project
from r1g1-4 and create the r1g1-4 tag.  
Use the svn\_load\_dirs.pl
helper script for this as described in the Red Book.  In the 
commands below the environment variable R1G14 points to a ``cleaned''
version of the ade/projects/ade directory as installed in 
accordance with the ADE flowtag.  (By ``cleaned'' I mean that 
we have removed files and directories we don't want under version
control, like the .svn directories.)

\begin{verbatim}
export URL=http://swdavison.dyndns.org:8080/awips2_repo
export SLDDIR=/home/swenv/swenv_progs/contrib/client-side/svn_load_dirs
export SLD=$SLDDIR/svn_load_dirs.pl
export R1G14=$HOME/ade.r1g1-4.cleaned
$SLD -t r1g1-4 $URL/vendor/ade current $R1G14
\end{verbatim}

\subsubsection{Prepare for the Merge}

Before proceeding, make sure that the trunk is up to date.
That is, reintegrate developer branches into the trunk.
(See the section titled ``How To Deliver Code to the Trunk.'')
Developers who do not reintegrate their branches into the trunk
at this stage will have a lot more work to do to deliver their
code later on.

As usual when reintegrating development branches, delete them after
the reintegration.

\subsubsection{Merge Vendor Changes into Working Copy of Trunk}

Make a working copy of the trunk.

\begin{verbatim}
svn checkout $URL/ade/trunk trunk_wc
\end{verbatim}

Merge the differences between the r1g1 tag and the current branch
(which is up to date with r1g1-4) into the working copy of the trunk.

\begin{verbatim}
svn merge    $URL/vendor/ade/r1g1     \
             $URL/vendor/ade/current  \
             trunk_wc
\end{verbatim}

\subsubsection{Resolve Conflicts, Test, Commit}

Resolve any conflicts between their changes and our changes.
This, of course, is the hard part.  Test to make sure everything 
works.

Finally, commit the working copy back into the trunk.

\begin{verbatim}
cd trunk_wc
svn commit -m"Merging Raytheon delivery r1g1-4 into trunk"
\end{verbatim}








\section{Using the Software Development Environment}

How to use Subversion and Trac (oriented toward devlopers,
not administrators).  ADE setup defined by Raytheon flowtag.  
Include by reference.  All Java development takes place within the 
Eclipse/ADE environment.

How to use the SDE software.  Step by step instructions for check 
in, check out, merge, deploy.  Generally organized to support the 
software development process section.

Perhaps incorporate here the HowToWiki from the NCLADT wiki 
(\url{https://collaborate.nws.noaa.gov/trac/ncladt/wiki/HowToWiki}).  
Some of it should be in the software development process 
section.  Example: Create a wiki 
page if you create an app (or new user-level function).  The wiki
has the potential to be a powerful communications tool if we 
can define good policies early and stick to them.  Need to 
learn some lessons from the NCLADT.  Note also that Trac provides a specific
mechanism for making inter-wiki links.  
\url{https://collaborate.nws.noaa.gov/trac/ncladt/wiki/InterWiki}

TBDs:
\begin{itemize}
\item
Document templates
\item
HowTo initialize a ticket
\item
HowTo schedule ticket to a build
\item
HowTo assign ticket to developer
\item
HowTo attach documents to ticket
\item
HowTo add comments with ticket number to source code so that 
Subversion hooks interact properly w/ Trac to link tickets
to specific code revision numbers.
\item
Needs thinking, discussion about how to set up 
the ticket categories, life cycles, roles, in Trac.  We
really won't know for sure until we start doing it.
\end{itemize}

% ----- Working with Subversion

\subsection{Working with Subversion}

Source file: AiiDhSwEnUsSu.tex

Subversion is our version control system.  It is a widely
used free, open-source package.  There is lots of good
documentation for Subversion:
\begin{itemize}
\item
If you simply type ``svn'' at the command line you will get
usage information leading you into the help facility that
gets installed with Subversion.
\item
There is a good book available: ``Version Control with
Subversion.''  You can buy a printed copy of the book,
access an HTML version on line, or download it as a PDF to
refer to or print out.  In this Handbook and in web pages
discussing Subversion it is often referred to as the ``Red Book''.
Please see the section of this Handbook 
titled ``Appendix \ref{ReferencesAndResources}: References and Resources.''
for full bibliographic information and URLs.
\item
Because Subversion is so widely used you can often get a
quick answer to a question by Googling.
\end{itemize}

This section of the Handbook provides an overview of
Subversion.  It is not intended to substitute for any of
the information sources listed above.  You will find it
well worth while to
become familiar with those sources.  

Other subsections of
the Handbook will provide cookbook-type examples of how to
accomplish tasks that may crop up frequently in the AWIPS
development environment.  What follows in this section is
a quick, high-level summary to give you some context when
you read the ``How To'' sections for specific tasks.

We can think of Subversion as consisting of three
components:
\begin{itemize}
\item
Data base files containing the source code that Subversion is
keeping track of.  Those files are on the server machine.  It
is best to consider those files a black-box implementation 
detail.  There will be little further discussion of them in
this document.  It is useful to remember, however, that a
source code file under Subversion configuration management
is not associated with any particular file in the Subversion
repository.  It is a set of records in one or more data
base cache files.
\item
Various server-side utility programs and API-accessible
functions.  The Subversion API is out of scope for this
document.  Svnadmin and svnlook are examples of server 
side utilities.  You can find examples of their usage in
the section titled ``Install Subversion,'' but they are
rarely needed by ordinary users and will not be discussed
further in this section.
\item
A server program.  Discussed in this section.
\item
One or more client programs.  Those must be installed on
the user's workstation.  Discussed in this section.
\end{itemize}


\subsubsection{Server}

Subversion comes with its own server (called svnserve),
but we do not use that.  Instead, we use the Apache web
server (httpd) as the subversion server. Httpd is
configured to call Subversion plugins when you access the
Subversion repository via http.  For example, if you enter
the following URL into a web browser you will see the
AWIPS2 repository:
\begin{verbatim}
    http://swdavison.dyndns.org:8080/awips2_repo
\end{verbatim}


\subsubsection{Client}

The AWIPS project uses several client programs for Subversion:
\begin{itemize}
\item
As indicated above, any web browser can serve as a
subversion client.
It provides a quick
and easy way to access the ``head'' revision of the software
under control.  (The head revision is the most recent
revision.)  But that's all you can do with the web
browser.  With the web browser you cannot edit code and
you cannot see old versions.
\item
Trac incorporates a Subversion client.  At the top of most
Trac web pages you will see a button labeled ``Browse
Source.``  That will take you into the repository.  It
only provides read access, but you can see old revisions,
commit comments, and other information.
\item
The version of Eclipse delivered by Raytheon as part of
the ADE includes a Subversion client packaged as a plugin
called Subclipse.  That allow you to use the Eclipse GUI
to access the repository, including checking code in and
out and other capabilities.
\item
Subversion provides a command line client called svn.  You
will see examples of svn usage in the rest of this section.
\end{itemize}


\subsubsection{Checkout and Commit (Checkin)}

Unlike other commonly used revision control systems like
CVS or RCS, Subversion does not (in the course of routine
usage) lock source code files.  If you execute the following
commands:
\begin{verbatim}
    mkdir ~/workarea.tmp
    cd ~/workarea.tmp
    export REPO_URL=http://swdavison.dyndns.org:8080/awips2_repo
    export HBK_URL=$REPO_URL/AiiDh/branches/sandbox
    svn checkout $HBK_URL
\end{verbatim}

you get a copy of the source code for this Handbook in the
workarea.tmp/sandbox directory.  In Subversion documentation
the workarea.tmp/sandbox directory is called a ``working
copy.''
On the server side, the code is
not locked.  The only record of the access will be in the
Apache httpd log.  Subversion will keep no record of it.
Other developers can check it out without any difficulty,
and without any notification that you have checked it out.
This is an intentional design feature of Subversion.  The
reasoning behind it is discussed in the Red Book.

(The checkout procedure
described above involves a lot of typing.  In actual
routine usage, as those experienced with development on
Unix platforms are aware, most of that typing would be
taken care of by environment variable and alias
definitions in the developer's environment setup files.)

Now you can edit the source code files for the handbook and,
when you are finished, you can check them in:
\begin{verbatim}
    svn commit -m"Example"
\end{verbatim}
(The above command assumes that your CWD is
~/workarea.tmp/sandbox.)


\subsubsection{Working Copies}

You will notice that the commit command did not contain
the repository URL.  How did svn know where to commit the
changes?  The answer is that the sandbox subdirectory,
created by the checkout command, contains metadata files
in a subdirectory called .svn.  Every source code directory
and subdirectory in a working copy contains a .svn
subdirectory.  Files in that directory contain 
information about the repository from which
the working copy was checked out.

One consequence of this arrangement is that only
directories, not individual files, can be checked out.
Individual files can be committed.

You can go ahead and execute the checkout and commit commands
above if you are interested.  The ``sandbox'' repository
directory, as its name indicates, is provided for
experimentation.  If it gets too messed up Subversion 
provides easy ways to restore it to a sane condition.

\subsubsection{Branches}
In Subversion jargon, the sandbox repository directory is 
called a branch.  In the AWIPS development environment
every developer working on a project has his or her
own branch.  The Subversion name of the Handbook project
is AiiDh.  That project has a subdirectory called branches.
Every developer working on the Handbook will have
a branch under that.  Indeed, you
might have more than one branch if you are working on
several tickets and want to keep the work separate.

Since Subversion stores all branches by deltas
(differences) they do not take up much space.  Since you
have your own branch, you can feel free to check your code
in whenever you choose without worrying about breaking
the build for other developers.  That way you have access to all your old
versions as well as taking advantage of the main
repository's backup facilities.


\subsubsection{Trunk}

Most repository projects contain a subdirectory called
``trunk''.  The trunk subdirectory contains reviewed,
``official'' code.  Branches are created by making copies
of the trunk, or of other branches.
Please be sure not to check code in to
the trunk without going through the proper reviews first.

If you accidentally check code into the trunk, please
notify the Repository Manager or the Development Manager
right away.  It's very easy to ``revert'' the trunk back to
its earlier state, so such accidents are not big deals as
long as they are promptly corrected.
But if they are left uncorrected then other changes may be
checked in on top of them.  After that has happened it's a
lot more work to disentangle the intentional and the
accidental updates.

As described above, when you commit code back to the
Subversion repository it goes back in to the part of the
repository from which it was checked out.  Since you
should not routinely check code in to the trunk, you
should avoid checking code out of the trunk.  If you want
a copy of the code in the trunk you should create a branch
and check the code out of that branch.  See the ``How To''
sections of this manual for a description of how to do
that.

Remember, by the way, that if you just want to look at the
code in the trunk you can do that through the read-only
interfaces provided by the web browser and by Trac.


\subsubsection{The Svn Client}

The svn client utility has a large number of subcommands.  The
Red Book has a reference section describing all of them.  Two
very useful subcommands are ``info`` and ``status``.

If you type ``svn info'' in a subdirectory created by an 
``svn checkout'' command, svn will look in the metadata directory 
(.svn) and provide information about the working copy,
including the repository branch from which it came.

If you type ``svn status'' in a Subversion working copy
directory, you will see a list of some of the files in the
directory.  Here is an example:
\begin{verbatim}
    $ svn status
    ?       AiiDh.out
    ?       AiiDh.pdf
    ?       globalRevisionNumber.tex
    ?       builddate.tex
    ?       AiiDh.log
    ?       AiiDh.toc
    ?       AiiDh.aux
    ?       AiiDh
    M       AiiDh.tex
\end{verbatim}

The files flagged by question marks are "unversioned files."  Those
are files that have never been checked in to subversion.  If you do
an ``svn commit'' operation those files will not be checked in.  In
this case we would not want those files to be checked in: They are
build products created by the build script and pdflatex processors
when the Handbook was built from source.  If you create a new file
that you want checked in, you must use the ``add'' subcommand.  For 
example:
\begin{verbatim}
    svn add NewClass.java
\end{verbatim}
That command will make svn update the metadata so that when you
later do an ``svn commit'' command the new Java source code file
will be added to the repository.

The file flagged with an ``M'' above has been modified since it
was checked out.  When you commit the directory the new version of
the file will go into the repository.


\subsubsection{Warning: Subversion Never Forgets}

Subversion never forgets.  That's good, because it makes it easy
to correct most accidental commit actions.  But it is {\em very} difficult
to {\em completely} remove anything from a Subversion repository.
There is a delete command.  That command creates a new version of the
repository without the deleted item.  But the item is still there in
previous versions of the repository.  So please take care
never to commit documents that should be private (password
files, personal files you do not want others to read, etc.)

Subversion stores source code files in compressed form as
data base entries.  The data base schema is a "black box" in
the sense that it is an implementation detail with no
official documentation.  Removing a file (or data from a
file)  completely would
require using SQL commands to delete certain
fields or rows, and updating whatever other tables might
refer to them.

It is very likely that almost all of the AWIPS code base will 
eventually be
released for public use.  But AWIPS does contain some restricted
code that the Government is permitted to use only under specific
licensing agreements.  We must take care not to publish any
repository which has ever held any of that code.




% ----- How To Edit this Handbook

\subsection{How To Edit this Handbook}
 
Source file: AiiDhSwEnUsHb.tex

This Handbook is maintained in the Subversion
repository as a set of \LaTeX\ files.
Members of the AWIPS development community are encouraged to
submit suggestions for additions, deletions, or changes to
the AWIPS Development Guidelines Working Group.  You can 
send suggestions to 
the document maintainer, whose email address appears in the
Introduction.
 
You can also edit the Handbook directly, but please
coordinate with the document maintainer before checking
anything in to the main trunk of the Subversion repository.

On the other hand, you can check anything you like in to
your own branch of the repository (how to create that branch
is described below).  Editing the Handbook is a good way to
get some familiarity with Subversion operations.
 
You will need package tetex-latex installed on your
computer.  If you want to build the html version of the
document you will also need package latex2html.  Both are
available from the Red Hat or CentOS repositories.

\begin{verbatim}
    # ----- Set an environment variable for the URL of the repository.
    export URL=http://swdavison.dyndns.org:8080/awips2_repo/AiiDh

    # ----- Make a branch of the Subversion for yourself.
    svn copy $URL/trunk $URL/branches/<yourNameHere> -m"New branch"

    # ----- Check the source code out of your new branch.
    mkdir ~/workarea
    cd ~/workarea
    svn checkout $URL/branches/<yourNameHere>

    # ----- Build the document.
    ./build.sh
\end{verbatim}

The build process generates several intermediate files and two
final build products: AiiDh.pdf and the AiiDh subdirectory.
You can inspect the first with any PDF reader (acroread, gs,
etc.).  The AiiDh directory contains the HTML version of the
Handbook, which you can inspect with any web browser.

Now you can edit the tex files with any text editor, rebuild
the document, and inspect the results.  At any time you can
check your work in:
\begin{verbatim}
    svn commit -m"checkpoint save"
\end{verbatim}

You may be able to learn what you need to know about
Latex markup by inspecting the source code files for the
Handbook and comparing them with the built result (PDF or
HTML).  If you want more information please see the 
section titled ``Appendix \ref{ReferencesAndResources}: References and Resources.''
% section titled Appendix \hyperref[ReferencesAndResources]{References and Resources}.
In particular, there
is link to a downloadable book from Wikibooks.  (Search for
``wikibooks.'')

See the section titled ``How To Deliver Code'' for instructions on 
how to merge your branch back in to the trunk so that the
Hudson build posts your new version on the team wiki.  But
please coordinate that delivery with the document maintainer
ahead of time.





% ----- How To Create a New Development Branch

\subsection{How To Create a New Development Branch}

Each developer should have his or her own development branch.
In some circumstances one developer might need more than one
branch.  Subversion maintains branches by keeping track of 
deltas from the trunk -- branches are cheap in terms of 
both disk space and processor load.

Create a branch whenever you need one.  Check code out from 
your branch when you need a working copy.  Commit code back
to the branch to save your unfinished work (a ``checkpoint''
commit).  Avoid checking out code from the trunk as much as
possible.  Remember that a working copy checked out from the
trunk is at risk of being checked back in to the trunk.
In fact, the only time you really need to check out code
from the trunk is when you are about to make a final delivery
of code.

Delete your development branch after you finish each task.  Then
create a new branch when you need it.

To create a new branch:
\begin{verbatim}
    export ROOT=http://swdavison.dyndns.org:8080/awips2_repo/AiiDh
    svn copy $ROOT/trunk $ROOT/branches/<yourNameHere> -m"Creating new branch"
\end{verbatim}




% ----- How To Update a Branch from the Trunk

\subsection{How To Update a Branch from the Trunk}

Source file: AiiDhSwEnUsUb.tex

In the AWIPS development environment each developer has
his or her own development branch.  It is important that you
keep your development branch up to date with respect to the
trunk.  Otherwise your code might not work when it is
finally delivered into the trunk.  You would use the svn
merge command to do that.

(The svn commands used here assume that you are using
Subversion 1.5 or later.  Merging with earlier versions 
of Subversion is more complicated because the client does
not keep track of the version at which branches were 
created; you must supply that information in the
merge command.)

At any time you can do a ``dry run'' merge to see what
is available to be merged from the trunk into your
development branch:

\begin{verbatim}
    # ----- Check your development branch out into a working copy.
    export ROOT=http://swdavison.dyndns.org:8080/awips2_repo/AiiDh
    svn checkout $ROOT/branches/<yourNameHere>

    # ----- Dry run merge to see if there are new changes in the trunk.
    cd <yourNameHere>
    svn merge --dry-run $ROOT/trunk
\end{verbatim}

If there is no output from the dry run merge your branch is up to
date with respect to the trunk.  Your branch might have new code
that the trunk does not have, but the trunk has no new code that your
branch does not have.

If you do get output it means that your branch is out of date.
The output might look like:
\begin{verbatim}
    --- Merging r50 through r56 into '.':
    U    AiiDhTools.tex
\end{verbatim}

That would tell you that you do need to update your
branch.  The file AiiDhTools.tex in your working 
copy would be updated if you actually ran the merge.  
Then (as shown below) you will need to commit your
updated working copy to update your branch.

To execute the merge, leave out the dry-run option:
\begin{verbatim}
    svn merge $ROOT/trunk
\end{verbatim}
You will see the same output you saw for the dry run.

The merge operation has updated the working copy.  You would want to
inspect the updated files to see what changed, and you might want to
build and test the code before committing it.  When you are ready, the
commit is straightforward:
\begin{verbatim}
    svn commit -m"Updated branch from trunk."
\end{verbatim}
Now your branch is up to date with respect to the trunk.

(Note that there is also an svn subcommand called ``update.''  That
subcommand updates your working copy from your development branch.
In development environments where many developers share the same
branch the svn update command is frequently used.  In the AWIPS
environment, where only one developer commits code to a given 
development branch, it will not be used so often.)




% ----- How To Deliver Code to the Trunk

\subsection{How To Deliver Code to the Trunk}

Source file: AiiDhSwEnUsDc.tex

The first step is to make sure that your development branch is 
up to date with respect to the trunk.  Please see the section
titled ``How To Update a Branch from the Trunk'' and carry out
those steps.

Next, make sure that both the trunk and your development branch
build and pass the regression tests.

The procedure for the delivery merge itself is as follows:

\begin{verbatim}
    # ----- Prepare a log message for the delivery commit.  You might
    # -----    start by collecting all the short commit messages you
    # -----    wrote while working.  You should also include any 
    # -----    relevant ticket numbers.
    export ROOT=http://swdavison.dyndns.org:8080/awips2_repo/AiiDh
    export BRANCHURL=$ROOT/branches/<yourNameHere>
    svn log --stop-on-copy $BRANCHURL > ~/commitmessage.tmp
    # ----- Edit the temporary file to create a commit message.
    vi ~/commitmessage.tmp

    # ----- Check out a new working copy from the trunk.
    export TRUNKURL=$ROOT/trunk
    svn checkout $TRUNKURL

    # ----- Merge your development branch into the new working
    # -----    copy of the trunk that you just created.
    cd trunk
    svn merge --reintegrate $BRANCHURL

    # ----- Build and test the newly-merged working copy.

    # ----- Commit the merged working copy back to the trunk.
    svn commit --file ~/commitmessage.tmp

    # ----- Clean up.  You don't need that working copy from the 
    # -----    trunk and it's dangerous to leave it around -- if
    # -----    you forgot it was from the trunk and edited and 
    # -----    committed you would have unwanted changes in the
    # -----    trunk.
    cd ..
    rm -rf trunk

    # ----- More cleanup.  Because of the reintegrate option on the
    # -----    merge command your development branch is no longer 
    # -----    usable.  Delete it.
    svn delete $BRANCHURL -m"Removing reintegrated branch."
\end{verbatim}

Although you just deleted your development branch it isn't really 
gone.  What you really did was to create a new repository version
that does not have that branch.  You could still recover the branch
by specifying an earlier version number in a checkout command.

If you need to create a new development branch for your next
task please see the section
titled ``How To Create a New Development Branch.''









