
\subsection{How To Update a Branch from the Trunk}

Source file: AiiDhSwEnUsUb.tex

In the AWIPS development environment each developer has
his or her own development branch.  It is important that you
keep your development branch up to date with respect to the
trunk.  Otherwise your code might not work when it is
finally delivered into the trunk.  You would use the svn
merge command to do that.

(The svn commands used here assume that you are using
Subversion 1.5 or later.  Merging with earlier versions 
of Subversion is more complicated because the client does
not keep track of the version at which branches were 
created; you must supply that information in the
merge command.)

At any time you can do a ``dry run'' merge to see what
is available to be merged from the trunk into your
development branch:

\begin{verbatim}
    # ----- Check your development branch out into a working copy.
    export ROOT=http://swdavison.dyndns.org:8080/awips2_repo/AiiDh
    svn checkout $ROOT/branches/<yourNameHere>

    # ----- Dry run merge to see if there are new changes in the trunk.
    cd <yourNameHere>
    svn merge --dry-run $ROOT/trunk
\end{verbatim}

If there is no output from the dry run merge your branch is up to
date with respect to the trunk.  Your branch might have new code
that the trunk does not have, but the trunk has no new code that your
branch does not have.

If you do get output it means that your branch is out of date.
The output might look like:
\begin{verbatim}
    --- Merging r50 through r56 into '.':
    U    AiiDhTools.tex
\end{verbatim}

That would tell you that you do need to update your
branch.  The file AiiDhTools.tex in your working 
copy would be updated if you actually ran the merge.  
Then (as shown below) you will need to commit your
updated working copy to update your branch.

To execute the merge, leave out the dry-run option:
\begin{verbatim}
    svn merge $ROOT/trunk
\end{verbatim}
You will see the same output you saw for the dry run.

The merge operation has updated the working copy.  You would want to
inspect the updated files to see what changed, and you might want to
build and test the code before committing it.  When you are ready, the
commit is straightforward:
\begin{verbatim}
    svn commit -m"Updated branch from trunk."
\end{verbatim}
Now your branch is up to date with respect to the trunk.

(Note that there is also an svn subcommand called ``update.''  That
subcommand updates your working copy from your development branch.
In development environments where many developers share the same
branch the svn update command is frequently used.  In the AWIPS
environment, where only one developer commits code to a given 
development branch, it will not be used so often.)


