
\chapter{Coding and Documentation Guidelines}


The team has found that the following publications contain 
useful descriptions of best practices.  We recommend that 
developers read these and observe the principles they
recommend.

TBD: List of publications.  Davison thinks a short list is
best.  Like two at the most.  Dave Plummer and Scott Jacobs 
maintain a small library for their development team.  One of
the books was Head First Design Patterns.  Can't remember
the others.  I suggest we identify one or two books that we're all 
willing to read ourselves and recommend it (or them) here.
And maybe get a few copies that developers can borrow.


When writing and reviewing code, keep in mind that it should
be comprehensible, maintainable, testable, and usable.

Comprehensible:
\begin{itemize}
\item
Write header comments (aka preamble or prolog) for every file and every
function.  The header comment should
explain what the code does and how it fits in to the larger system.
Assume as little knowledge as possible, but achieve brevity
and avoid duplication by referring to other project
documentation, including other source code, as needed.
See the example code for the formats for file and function 
header comments.
\item
Make your code self-documenting.  Use descriptive variable
names.  Avoid magic constants.
\item
Provide brief internal comments as needed.  
\item
Use well-known patterns whenever possible, identify them
by name in comments and by class and method naming
conventions.
\end{itemize}


Maintainable:
\begin{itemize}
\item
Search out and use existing utility libraries.
Don't duplicate functionality even if you think you are providing
an improved implementation.
section titled ``Development Case Studies, Infrastructure Capabilities,
and Interfaces'' for guidance on access to
AWIPS infrastructure.
\item
Include javadoc (or pydoc) comments even for private functions.  
\item
Every file should include a revision history in the file
header comment.  (TBD: Do we really want to require this?  We now 
have Subversion
commit comments.  This is additional work for the programmer
that may not be necessary.)
\item
Define variables in the narrowest possible scope.  
\item
Strive for simplicity.  Each function should do one thing only.
\end{itemize}

Usable:
\begin{itemize}
\item
Include javadoc and pydoc comments.  Make sure that interfaces are well
documented.  
\item
Trap bad calls and error conditions.  Throw informative exceptions, 
use debug logging as appropriate.
\end{itemize}

Testable:
\begin{itemize}
\item
Include Junit tests for Java classes and methods.
\item
Provide tests that can be run unattended (automated
testing).  Keep in mind that changes to code
that you've never seen or thought about (class ancestors,
utility functions) can break your code.  Obviously your 
code needs to be tested before checkin, but it also needs
to be retested when other code is checked in.  The only
practical way to achieve that is by automated testing.
TBD: Provide a small buildable example to show the preferred
or required autotesting implementation.  Developers should
be able to check out the example code, build it in their
workspace, see the output showing tests passing or failing.
\end{itemize}

See the code in the Appendix section titled ``Exemplary
Source Code'' for examples of the required 
coding practices.  They can be used as templates.  The files 
are available from the Subversion repository.

TBD: Exemplary files.  What do we need?
\begin{itemize}
\item
Java class
\item
Python script
\item
Bash shell script
\item
Others?
\end{itemize}

TBD: What else should go in this section?
\begin{itemize}
\item
Package naming convention
\item
Directory structure
\end{itemize}




