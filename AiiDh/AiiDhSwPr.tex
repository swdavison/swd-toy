
\chapter{Software Development Process}



The software development process is a set of procedures to ensure that:
\begin{itemize}
\item
subsystem design is well-documented;
\item
significant design decisions are documented;
\item
design and code satisfy approved requirements;
\item
design and code conform to project standards;
\item
code is saved in a configuration management system;
\item
design and code are accessible to other team members;
\item
there is adequate opportunity for formal and informal review, 
discussion, and feedback at appropriate stages in the software 
development process.
\end{itemize}


For brevity and clarity the work flow description that follows
refers to Trac (the issue tracking software the team is using)
and Subversion (the version control software).  Other software
could support the same work flow.  For specific instructions
on how to use Trac and Subversion to execute the steps in
the work flow, please see the section titled ``Using the 
Software Development Environment.''

\begin{quotation}
TBD:
(This is an elaboration of material from the NCEP wiki.  Contains
some inferences.  NCEP review and comment needed.)
\end{quotation}


Work Flow:
\begin{itemize}
\item
Someone initializes a document (hereafter called a ticket) in the 
issue tracking system.  It serves the role of a functional requirements 
statement for a chunk of work.  It might be an enhancement request or 
it might be a bug report.  The ticket serves as a home for 
documentation relating to that particular chunk of work.  Anyone can 
initially create a ticket.  Nothing happens to it until a 
Development Manager assigns it to a build.  Each ticket has a unique
number which may be incorporated in code comments to support 
traceability.  If the ticket is derived from some other requirements
list or specification, the source requirement should be unambiguously
identified in the ticket.
\item
Development Manager schedules the ticket for a specific build.
\item
Development Manager assigns the ticket to a developer.  That 
assignment is documented in the ticket, and provides the developer 
with a particular role with respect to the ticket.
\item
Developer does analysis and design.  Developer writes up and adds 
to the ticket:
\begin{itemize}
\item
Detailed requirements statement; 
\item
High-level design, including alternatives and justification for 
selection of alternative to be implemented; 
\item
Detailed design;
\item
Test plan.
\end{itemize}
There may be one or more review stages between addition to the 
ticket of the above artifacts.  Depending on the expected size 
of the chunk of work some of the artifacts may be combined.  
Artifacts may be new text typed added to the ticket or 
documents attached to the ticket.  Please see the section
titled ``Using the Software Development Environment'' for
document templates for the artifacts.  (TBD:  Create the
templates.  Put the appropriate directory paths for the
templates in to the Using the SDE section.)
\item
Developer does the development work.  She has her own branch 
within the configuration management system so she can check 
code in and out without worrying about breaking the build for 
others.  Periodically she merges code from the trunk into her 
branch so she is building against recently-delivered code.

The code she writes should be consistent with the design
documents listed previously.  If necessary, the developer
should update the design documents.  In that case the
Development Manager must be notified.  The Development
Manager may schedule a review.
\item
Developer completes the work.  She writes up a delivery statement 
describing what she has done, adds it to the ticket.  Schedules a review.
\item
Review is held.  Assuming code passes review, the developer 
commits it to the main trunk of the source code control system.  
She includes the ticket number in her check-in comment.
As part of the commit process, the source code control system automatically 
invokes the issue tracking system to link the updated 
source code versions to the ticket  and to make 
sure that the developer has the role to check 
code in against that ticket.  
\end{itemize}

TBD:  Need to say something about the nature of the reviews:
\begin{itemize}
\item
How flexible?  Full-bore, sit-down reviews in a meeting
room in some cases, maybe circulation of email in others.
\item
By whom?  Peers identified by developer?  Peers assigned
by the DM?
\item
What degree of accountability?  Should reviewers sign
off on documents?
\item
Reviews useful not just for QC, but also to keep team
members abreast of what others are doing and how they are
doing it.
\item
Need to keep in mind that reviews can be real time-eaters.
A small group (like two reviewers) specifically assigned, 
held accountable, and 
allowed time probably better than a big group.  
\end{itemize}



